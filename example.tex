\documentclass{article}
\usepackage[noend]{distalgpseudocode}
\usepackage{algorithm}
\begin{document}

\begin{algorithm}[t]
	\caption{Send, receive and eavesdrop example}
	\label{alg:algo1}
	\begin{algorithmic}[1]
	
	\Function{fn}{}
	    \LineComment{Use to broadcast a message}
	    \State \Broadcast{MessageName}{parameters}
	    
	    \LineComment{Use to unicast a message to a specific target}
        \State \SendTo{MessageName}{parameters}{target}
	\EndFunction
	
	\Statex
	
	\LineComment{Use to receive a message}
	\Event{MessageName}{parameters}
        \State \ldots
    \EndEvent
    
    \Statex
	
	\LineComment{Use to receive a message and also be aware of the source}
	\ReceiveFrom{MessageName}{parameters}{source}
        \State \ldots
    \EndReceive
    
    \Statex
    
    \LineComment{Use to eavesdrop a message}
    \Snoop{MessageName}{parameters}
        \State \ldots
    \EndSnoop
    
    \Statex
    
    \LineComment{Use to eavesdrop a message and also be aware of the source}
    \SnoopFrom{MessageName}{parameters}{source}
        \State \ldots
    \EndSnoop

	\end{algorithmic}
\end{algorithm}


\begin{algorithm}[t]
	\caption{Event signalling and triggering}
	\label{alg:algo2}
	\begin{algorithmic}[1]
	
	\Function{fn}{}
	    \LineComment{Use to signal that an even has occurred}
	    \Signal{EventName}{parameters}
	\EndFunction
	
	\Statex
	
	\LineComment{Use to handle an event}
	\Event{EventName}{parameters}
        \State \ldots
    \EndEvent
    
    \Statex
	
	\LineComment{Some events may not occur when a signal happens, but may be specified with a predicate that must be true for them to be triggered. This is inspired by Guarded Command Notation}
	\CondEvent{ConditionalEventName}{predicate}
        \State \ldots
    \EndCondEvent

	\end{algorithmic}
\end{algorithm}

\begin{algorithm}[t]
	\caption{Timers}
	\label{alg:algo3}
	\begin{algorithmic}[1]
	
	\Function{fn}{}
	    \Let{TimerName}{$\hourglass$} \Comment{Initialise a timer}
	    \State \Now{} \Comment{Get the current time}
	    \State \StartTimer{TimerName}{10} \Comment{Start the timer to expire in 10 time units}
	    \State \StartTimerAt{TimerName}{10}{5} \Comment{Start the timer to expire in 10 time units after 5}
	    \State \IsTimerRunning{TimerName} \Comment{Is the timer running?}
	    \StopTimer{TimerName} \Comment{Stop the timer}
	\EndFunction
	
	\Statex
    
    \LineComment{Use to handle a timeout}
	\Timeout{TimerName}
	    \Let{t}{\TimerFiredAt{TimerName}} \Comment{When was this timer fired}
        \State \ldots
    \EndTimeout
    
    \Statex
	
	\LineComment{Use to handle a timeout and get the time at which the timeout occurred}
    \TimeoutAt{TimerName}{time}
        \State \ldots
    \EndTimeout

	\end{algorithmic}
\end{algorithm}

\end{document}
